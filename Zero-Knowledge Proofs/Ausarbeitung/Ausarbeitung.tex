

\documentclass {article}

\usepackage[backend=bibtex,style=alphabetic]{biblatex}
\usepackage[german]{babel}
\bibliography{lit}
\usepackage{csquotes}
\MakeOuterQuote{"}

\title{Zero-Knowledge Proofs}
\author{Max Ostermann}
\date{28.05.2019}

\begin{document}

\maketitle
\pagenumbering{gobble}
\newpage
\pagenumbering{arabic}

\section{Abstract}


Diese Ausarbeitung ist eine Einf\"uhrung in Zero-Knowledge Proofs.
Nach der Einf\"uhrung anhand von Alibabas H\"ohle und der Erkl\"arung der grundlegenden Definitionen werden die Protokolle von Zero-Knowledge Proofs untersucht.
Bei dieser Betrachtung stellt man zum einen fest, dass die arithmetischen Operationen für ehrliche Beweisteilnehmer einen sehr geringen Rechenaufwand haben und zum anderen, dass f\"ur alle Sprachen in NP ein Zero-Knowledge Proof existiert. Trotzdem f\"allt auf, dass selbst bei leicht berechenbaren Operationen die reine Anzahl an Interaktionen eine potenzielle Limitierung von Protokollen darstellt.
Dank der zus\"atzlichen M\"oglichkeiten, die sich durch non-interactive Zero-Knowledge Proofs ergeben, sind Zero-Knowledge Protokolle nicht nur zur Authentifizierung geeignet, sondern lassen sich auch für Signaturen verwenden.
Unter Ber\"ucksichtigung der Schw\"achen, wie zum Beispiel der "man-in-the-middle" Attacke, stellt sich heraus, dass Zero-Knowledge Proofs am besten als Teilprotokolle verwendbar sind. Dies zeigt sich unter anderem in neuen Blockchain Anwendungen, bei denen die Integrität der einzelnen Transaktionen durch ein eigenes Protokoll sichergestellt wird und Zero-Knowledge Proofs nur zum Schutz der Privatsph\"are der Nutzer dienen.
\\


\newpage

\tableofcontents

\newpage 

\section{Einf\"uhrung}

In der Kryptographie geht es häufig darum, eine geheime Botschaft sicher zu übermitteln, ohne dass Dritte Zugriff auf ihren Inhalt haben. Möchte man stattdessen nur beweisen, dass man dieses Geheimnis kennt, ohne Informationen preiszugeben, bieten sich Zero-Knowledge Proofs an. Mit Zero-Knowledge Proofs beweist ein Beweiser einer verifizierenden Person sein Wissen über ein Geheimnis, ohne dass diese zusätzliche Informationen über das Geheimnis erlangen kann.\\
Dies wird im Laufe dieses Kapitels anhand zweier Beispielen verdeutlicht. Anschließend werden die grundlegenden Definitionen(3) und mit Hilfe derer der Begriff der Simulierbarkeit(4) erklärt. 
Darauf folgt eine kurze \"Ubersicht über einige Protokolle und es wird die Existenz von Zero-Knowledge Proofs zu Sprachen in NP gezeigt. Unter Betrachtung der neuen M\"oglichkeiten, die sich mit non-interactive Zero-Knowledge Proofs(6) bieten , werden die aktuellen Anwendungen(7) von Zero-Knowledge Proofs erl\"autert und abschließend noch deren Schw\"achen betrachtet.
\\

Schon im Jahr 1535 fand der italienische Mathematiker Niccolo Tartaglia die erste Verwendung für Zero-Knowledge Proofs\cite{BSW}. Er entdeckte eine Lösungsformel für Polynome 3. Grades und wollte dies beweisen, ohne - aus Furcht ein anderer Mathematiker würde es als seine Entdeckung verkaufen - diese Formel preiszugeben. Dazu bat er einen Kollegen seines Fachs um Aufgaben, welche er ihm wieder zukommen lie\ss{}. Mit seiner L\"osungsformel konnte er die Aufgaben korrekt l\"osen, so beweisen, dass er im Besitz der korrekten Formel war und anschlie\ss{}end als seine Entdeckung vermarkten. \\

Wie man hier erkennen kann, dienen Zero-Knowledge Proofs dazu, als Beweiser, auch Prover/ P genannt, eine
verifizierende Person, auch Verifier/ V genannt, von der Kenntnis eines Geheimnisses zu überzeugen, ohne dieses preiszugeben. \\ 

Um die meisten Eigenschaften von Zero-Knowledge Proofs zu veranschaulichen, stellten Quisquater und Guillous in "How to explain Zero-Knowledge Protocols to your Children"\cite{GQ89} das Konzept von Alibabas H\"ohle vor.
Alibabas H\"ohle hat einen Vorraum, hinter dem ein Rundgang liegt. In der Mitte dieses Ganges ist eine magische T\"ur. Diese T\"ur l\"asst sich nur mit Kenntnis der geheimen Passphrase \"offnen,
Alice kennt dieses Geheimnis, Bob jedoch nicht.
Nun m\"ochte Alice Bob beweisen, dass sie dieses Geheimnis kennt, ohne Bob eben jenes zu offenbaren. Also befiehlt sie Bob vor der H\"ohle zu warten, w\"ahrend sie selbst in einen der beiden Eing\"ange des Rundgangs hineingeht. Bob betritt daraufhin den Vorraum und ruft aus welcher Seite Alice den Rundgang verlassen soll. Wenn Alice den Gang durch diese Seite betreten hat, dreht sie um und kommt zu dieser Seite wieder heraus. Hat sie jedoch den Gang von der anderen Seite betreten, muss sie nun die Passphrase verwenden, um durch die T\"ur gehen zu k\"onnen und in dem von Bob gewünschten Ausgang zu erscheinen. Nach einer einzigen Durchf\"uhrung dieses Spiels ist Bob zurecht noch nicht von Alice Wissen \"uberzeugt, da es sich um Zufall handeln k\"onnte. Doch vertraut er Alice nach jeder neuen Runde etwas mehr und ist nach einigen Runden \"uberzeugt, dass Alice die geheime Passphrase kennt, ohne dass Bob neue Informationen \"uber die Passphrase erhalten hat. \\ 

Nun m\"ochte Bob dieses Ph\"anomen mit der Welt teilen und zeichnet seine Sicht mit einer Kamera auf. Doch stellt er fest, dass niemand ihm glauben würde, da er das Video auch einfach selber f\"alschen k\"onnte, indem er sich entweder mit einem Freund abspricht, in welcher Reihenfolge er die Ausg\"ange ausruft oder die fehlerhaften Ergebnisse einfach aus dem Video herausschneidet. 
Stellt man die Videos, die Alice oder Bobs Freunde zeigen, nun gegen\"uber, zeigt sich, dass sich diese kaum voneinander unterscheiden und ein Beobachter dieser Videos keine M\"oglichkeit hat abzuw\"agen, von wem er denn das Geheimnis erlangen kann.
Diese Eigenschaft wird unter "Simulierbarkeit"(5) n\"aher betrachtet und Alibabas H\"ohle wird in den danach folgenden Abschnitten zur Veranschaulichung entsprechend erweitert.


\section{Grundlagen}
\subsection{Definitionen}
\subsection{Beweissysteme}
\section{Sicherheitsbeweise mit Hilfe der Simulierbarkeit}
\subsection{Honest-Verifier Simulator}
\subsection{Black-Bóx Simulator}
\section{Protokolle}
\subsection{Graphisomorphismen}
\subsection{Quadratische Reste}
\subsection{Graphfärbbarkeit}
\subsection{Existenz von Zero-Knowledge Proofs zu jeder Sprache in NP }
\section{Non-interactive Zero-Knowledge Proofs}
\subsection{Motivation}
\subsection{Fiat-Shamir Heuristik}
\section{Moderne Anwendungen}
\subsection{Authentifizierung}
\subsection{Blockchain}
\subsection{Signaturen}
\section{Angriffsvektoren}
\subsection{Man-in-the-middle}
\subsection{Koordinierte Angreifer}
\section{Fazit}

\newpage
\section{Quellenverzeichnis}

\nocite{BG89}
\nocite{BM89}
\nocite{GMR85}
\nocite{GO94}
\nocite{KZKP}
\nocite{PrSa14}
\nocite{RS91}
\nocite{BHHW}

\printbibliography


  


\end{document}


